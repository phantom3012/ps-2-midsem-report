\chapter{Product walkthrough} % Main chapter title

\label{ch:walkthrough}

\lhead{Chapter 2. \emph{Product walkthrough}}

\section{Hardware setup}

In the overarching architecture of the wBMS, the fundamental structure comprises two primary components: a device to monitor battery data and a device that collects the relevant data (battery monitor sensors).

To facilitate seamless utilization of this collected data by client microcontrollers, a specialized software tool is employed which is designed to run parallel with compiled applications of said microcontroller that expose this data to be easily monitored via external methods.

It is important to recognise that the integrity of transmitted data may be vulnerable to numerous errors along its path due to the complex system architecture. Keeping in mind the various faults that could occur while transmitting data from one end to another, various safety guidelines have been agreed upon to be followed by the multiple components in the system.

The safety guidelines outline certain measures that must be adhered to. These techniques are intended to guarantee that, even in the presence of defects, the system can continue to warn relevant parts about the presence of a fault while supporting the ultimate goal of adhering to set functional safety requirements

\section{Software setup}

Given the multiple pieces of hardware present in this setup, there has to be relevant software that runs on each of them.

The sensor that essentially gathers the battery data is given certain instructions (through software) at certain intervals of time, which are then collected and ultimately reported to the client microcontroller that requests the data.

A methodical approach is used, with these instructions serving as a key instrument, to ensure that the monitoring sensor complies with the safety regulations.

In order to ensure the utmost stability of the sensor and compliance with established safety protocols, two potential approaches can be employed. Firstly, the sensor can be subjected to the injection of erroneous and unreliable instructions. Alternatively, the sensor's output can be intentionally corrupted, and the resulting responses at the monitoring end can be compared to a pre-existing baseline. This comparative analysis serves to evaluate the sensor's performance.

To aid in this process of cleanly monitoring the sensors, multiple test frameworks for the various components in the system have been designed and tested.